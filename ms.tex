\documentclass[letterpaper, 11pt]{article}
\usepackage{graphicx}
\usepackage{natbib}
\usepackage[left=2.5cm,top=2.5cm,right=2.5cm]{geometry}
\usepackage{parskip}
\usepackage{amsmath}
\usepackage[utf8]{inputenc}

\renewcommand{\topfraction}{0.85}
\renewcommand{\textfraction}{0.1}

\newcommand{\hyperparams}{\boldsymbol{\alpha}}
\newcommand{\xx}{\mathbf{x}}

\title{What is MaxEnt?}
\author{Brendon J. Brewer}

\begin{document}
\maketitle



Relative entropy
\begin{eqnarray}
H(\mathbf{p}; \mathbf{q}) &=& -\sum_i p_i \log\left(\frac{p_i}{q_i}\right) 
\end{eqnarray}

\section{Testable Information}

\section{Where does testable information come from?}

Example about hearing ``sí/si'' as an answer to a question,
and inferring the language (Spanish or French).

The joint prior is:

\begin{eqnarray}
\begin{array}{|c|c|c|}
\hline
	&	S	& \bar{S}\\
\hline
H_1 & 0.25  & 0.25\\
H_2 & 0.05 & 0.45\\
\hline
\end{array}
\end{eqnarray}

The marginal probabilities for $H_1$ and $H_2$ are both 0.5, and the
marginal probabilities for $S$ and $\bar{S}$ are 0.3 and 0.7 respectively.


\section{MaxEnt and Bayesian Inference}

\section{MaxEnt and Hierarchical Bayesian Models}

\begin{thebibliography}{}
\end{thebibliography}

\end{document}

